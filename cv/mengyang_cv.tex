% Copyright 2019 Matthew Felix Yuan
%
% This work may be distributed and/or modified under the
% conditions of the LaTeX Project Public License, either version 1.3
% of this license or (at your option) any later version.
% The latest version of this license is in
%   http://www.latex-project.org/lppl.txt
% and version 1.3 or later is part of all distributions of LaTeX
% version 2005/12/01 or later.
%
% This work has the LPPL maintenance status `maintained'.
% 
% The Current Maintainer of this work is Matthew Felix Yuan.
%
% This work consists of the files blank-template.tex and sample-resume.tex.



\documentclass[11pt]{article}
% helpful packages.
\usepackage{geometry}
\usepackage[hidelinks]{hyperref}
\usepackage{indentfirst}
\usepackage{enumitem}
\usepackage[T1]{fontenc}
%\usepackage{hanging}
% global formatting tweaks.
\pagestyle{empty}
\geometry{margin=0.75in}
\setlength\parindent{9pt}
\setlist{nosep}
\renewcommand\labelitemi{--}
% new commands!
\newcommand{\name}[1]{\begin{center}\section*{\Huge #1}\end{center}}
\newcommand{\topinfo}[1]{\begin{center}\vspace{-0.2cm}#1\vspace{-0.2cm}\end{center}}
\newcommand{\resumesection}[1]{\vspace{-0.2cm}\section*{#1}\vspace{-0.2cm}\hrule\vspace{0.2cm}}



\begin{document}

% \name is self-explanatory. \topinfo is for things like social media, email, phone number, address, etc.
\name{Mengyang Qiu}
\topinfo{122 Cary Hall, Buffalo, NY 14214}
\topinfo{\href{mailto:mengyang@buffalo.edu}{mengyang@buffalo.edu}}
\topinfo{\urlstyle{same}\url{https://mengyang.dev}}

% name your resume sections whatever you want!
\resumesection{Education}
% things written after \hfill will be right-aligned, which is where i usually put dates.
\textbf{University at Buffalo, The State University of New York}

\textbf{Ph.D., Communicative Disorders and Sciences} \hfill 2016--2021(Expected)

\quad Advisor: Brendan T. Johns, Ph.D.

\textbf{M.S., Computational Linguistics} \hfill 2018--2019

\quad M.S. Project: Artificial Error Generation with Fluency Filtering

\quad Advisor: Jungyeul Park, Ph.D.

\textbf{M.A., General Linguistics} \hfill 2013--2016

\quad M.A. Project: Verb Aspect and the Activation of Location Information

\quad Advisor: Jean-Pierre Koenig, Ph.D.

\textbf{B.A., Psychology \& Linguistics (Honors in Language and Cognition)} \hfill 2010--2013
% each \item is a point you want to say about the thing.
%\begin{itemize}
%	\item Description of thing.
%	\item More description of thing.
%\end{itemize}


% fill your resume with sections!
\resumesection{Honors and Awards}
\textbf{University at Buffalo, The State University of New York}

Mark Diamond Research Fund, \$2964 \hfill 2020--2021

Teaching Assistantship with Full Tuition Scholarship \hfill 2013--Present

GSA Conference Funding, \$550 \hfill 2018

New York State Graduate Student Employees Union Award, \$2202.50 \hfill 2015

Wolfgang W{\"o}lck Outstanding Graduating Senior Award, \$250 \hfill 2013


\resumesection{Publications}

\begin{itemize}[leftmargin=!,labelindent=!,itemindent=-15pt]
\setlength\itemsep{0.3em}

%\item[] \textbf{Qiu, M.} \& Johns, B. T. (under review). A distributional and sensorimotor analysis of noun and verb fluency.

\item[] \textbf{Qiu, M.} \& Johns, B. T. (2020). \href{https://rdcu.be/bYRm5}{Semantic diversity in paired-associate learning: Further evidence for the information accumulation perspective of cognitive aging}. \textit{Psychonomic Bulletin \& Review}, \textit{27}(1), 114-121.

\item[] \textbf{Qiu, M.} \& Park, J. (2019). \href{https://www.aclweb.org/anthology/W19-4408/}{Artificial error generation with fluency filtering}. \textit{Proceedings of the 14th Workshop on Innovative Use of NLP for Building Educational Applications}.

\item[] \textbf{Qiu, M.}, Chen, X., Liu, M., Parvathala, K., Patil, A. \& Park, J. (2019). \href{https://www.aclweb.org/anthology/W19-4425/}{Improving precision of grammatical error correction with a cheat sheet}. \textit{Proceedings of the 14th Workshop on Innovative Use of NLP for Building Educational Applications}.

\end{itemize}

\resumesection{Conference Presentations and Posters}

\begin{itemize}[leftmargin=!,labelindent=!,itemindent=-15pt]
\setlength\itemsep{0.3em}
\item[] \textbf{Qiu, M.} (2020, November). Using natural language processing for language sample analysis. Proposal accepted at the \textit{Annual Convention of the American Speech-Language-Hearing Association}, San Diego, CA (Convention canceled).

\item[] Sun, S. \& \textbf{Qiu, M.} (2020, November). Early production of animal vocabulary in Mandarin Chinese and American English. Proposal accepted at the \textit{Annual Convention of the American Speech-Language-Hearing Association}, San Diego, CA (Convention canceled).

\item[] \textbf{Qiu, M.} \& Johns, B. T. (2018, November). Memory searching pathway underlying verb fluency. Poster presented at the \textit{Annual Convention of the American Speech-Language-Hearing Association}, Boston, MA.

\item[] Min, H., Guo, L., Higginbotham, D. J., \& \textbf{Qiu, M.} (2018, November). How does writing medium influence the operation of writing processes and writing quality? Poster presented at the \textit{Annual Convention of the American Speech-Language-Hearing Association}, Boston, MA.

\end{itemize}

\resumesection{Research Experience}
\textbf{University at Buffalo, The State University of New York}

\textbf{Programmer} | PI: Thea Knowles, Ph.D. \hfill Spring 2020

\quad Clinical Applications of Speech Acoustics Lab

\quad Department of Communicative Disorders and Sciences

\textbf{Project Leader} | Supervisor: Jungyeul Park, Ph.D. \hfill Spring 2019

\quad Grammatical Error Correction Project, Department of Linguistics

\textbf{Graduate Research Assistant} | PI: Brendan T. Johns, Ph.D. \hfill 2016--2019

\quad Computational Language and Memory Lab

\quad Department of Communicative Disorders and Sciences

\textbf{Graduate Research Assistant} | PI: Christopher McNorgan, Ph.D. \hfill Summer 2017

\quad Computational Cognitive Neuroscience Lab, Department of Psychoplogy

\textbf{Undergraduate Research Assistant} | PI: Gail Mauner, Ph.D. \hfill 2012--2013

\quad Psycholinguistics Lab, Department of Psychoplogy

\textbf{Undergraduate Research Assistant} | PI: Paul Luce, Ph.D. \hfill Summer 2012

\quad Language Perception Lab, Department of Psaychology

\textbf{Undergraduate Research Assistant} | PI: Eon-Suk Ko, Ph.D. \hfill Summer 2011

\quad Child Language Development Project, Department of Linguistics

\resumesection{Teaching Experience}
\textbf{Teaching Assistant, University at Buffalo}

\textbf{Department of Communicative Disorders and Sciences} \hfill 2016--Present

\quad CDS 151: Intro to Speech-Language Pathology and Audiology

\quad CDS 564: Language Disorders in Adults

\quad CDS 480/687: Disorders of Memory

\textbf{Department of Linguistics} \hfill 2013--2015

\quad CHI 101 \& 102: First-Year Chinese

\resumesection{Professional Service}

\textbf{Reviewer} | 15th Workshop on Innovative Use of NLP for Building Educational Applications \hfill 2020

\textbf{Student Reviewer} | Journal of Speech, Language, and Hearing \hfill 2017

\quad Under the supervision of Ling-Yu Guo, Ph.D.

\resumesection{Technical Skills}

\textbf{Programming:} Python, Java, Javascript

\textbf{Statistics:} R, Matlab, SPSS

\textbf{Experiment:} jsPsych, PsychoPy/JS, JATOS, psiTurk

\textbf{NLP:} PyTorch, Keras, SpaCy

\textbf{Others:} \LaTeX, Git, Prolog, Praat

%\resumesection{Other Information}

%\textbf{Citizenship:} China

%\textbf{Permanent Residency:} Canada

%\textbf{Languages:} Mandarin (native), English (fluent), Cantonese (receptive)


\end{document}
